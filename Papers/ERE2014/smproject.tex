\documentclass[a4paper]{jpconf}
\usepackage{graphicx}
\usepackage{hyperref}

\newcommand{\soft}[1]{\texttt{#1}}

\begin{document}
\title{Tensor calculus with free software: \\
the SageManifolds project}

\author{Eric Gourgoulhon$^1$, Micha\l{} Bejger$^2$, Marco Mancini$^1$}

\address{$^1$ Laboratoire Univers et Th\'eories, UMR 8102 du 
CNRS, Observatoire de Paris, Universit\'e Paris Diderot,
92190 Meudon, France}

\address{$^2$ Centrum Astronomiczne im. M. Kopernika, ul. Bartycka 18,
00-716 Warsaw, Poland}

\ead{eric.gourgoulhon@obspm.fr}

\begin{abstract}
Abstract
\end{abstract}

\section{Introduction}

Computer algebra for general relativity (GR) has a long history, which started
almost as soon as computer algebra itself in the 1960s. 
The first GR software was \soft{ALAM} (for \emph{Atlas Lisp Algebraic Manipulator})
witten by R.A. d'Inverno in 1969, who used it to compute
the Riemann and Ricci tensors of the Bondi metric. The original calculations took Bondi and his collaborators 6 months to go. The computation with \soft{ALAM} took 4 minutes and yield to the discovery of 6 errors in the original paper \cite{Skea94}. 
Since then, many packages have been developed and the reader is refered to \cite{MacCa02}
for a review of computer algebra
systems for GR prior to 2002 and to \cite{KorolKS13} for a more recent review,
focussing on tensor calculus, as well as to the semi-exhaustive list of
tensor calculus packages at \cite{xact_links}.

%%%%%%%%%%%%%%%%%%%%%%%%%%%%%%

\section{Software for differential geometry}

Software packages for differential geometry and tensor calculus can be 
classified in two categories: 
\begin{enumerate}
\item Applications atop general purpose computer algebra systems; 
notable examples are 
\soft{xAct} \cite{Marti08,xAct}, \soft{Ricci} \cite{Ricci}, both
running atop \soft{Mathematica},
\soft{DifferentialGeometry} \cite{AnderT12,DiffGeom} integrated into \soft{Maple}, and \soft{Atlas 2}
\cite{Atlas2} for both \soft{Mathematica} and \soft{Maple}.
\item Standalone applications; recent examples are \soft{Cadabra} \cite{Peete07,Cadabra} (field theory),
\soft{SnapPy} \cite{SnapPy} (topology and geometry of 3-manifolds) and
\soft{Redberry} (tensors) \cite{BolotP13,Redberry}.
\end{enumerate}
All applications listed in the second category are free (open-source) software. In
the first category, \soft{xAct} and \soft{Ricci} are also free software, but
they run atop a proprietary product (\soft{Mathematica}). 



%%%%%%%%%%%%%%%%%%%%%%%%%%%%%%

\section*{References}
\begin{thebibliography}{10}
\bibitem{Skea94}
Skea J E F 1994 Applications of SHEEP {\it Lecture notes available at}
\url{
http://www.computeralgebra.nl/systemsoverview/special/tensoranalysis/sheep/}
\bibitem{MacCa02}
MacCallum M A H 2002 {\it Int. J. Mod. Phys. A} {\bf 17}, 2707 
\bibitem{KorolKS13}
Korol'kova A V, Kulyabov D S and Sevast'yanov L A 2013 {\it Prog. Comput. Soft.} 
{\bf 39}, 135
\bibitem{xact_links} 
\url{http://www.xact.es/links.html}
\bibitem{Marti08}
Martin-Garcia J M 2008 {\it Comput. Phys. Commun.} {\bf 179}, 597
\bibitem{xAct}
\url{http://www.xact.es}
\bibitem{Ricci}
\url{http://www.math.washington.edu/~lee/Ricci/}
\bibitem{AnderT12}
Anderson I M and Torre C G 2012 {\it J. Math. Phys.} {\bf 53}, 013511
\bibitem{DiffGeom}
\url{http://digitalcommons.usu.edu/dg/}
\bibitem{Atlas2}
\url{http://digi-area.com/Maple/atlas/}
\bibitem{Peete07}
Peeters K 2007 {\it Comput. Phys. Commun.} {\bf 15}, 550
\bibitem{Cadabra}
\url{http://cadabra.phi-sci.com/}
\bibitem{SnapPy}
Culler M, Dunfield N M and Weeks J R, SnapPy, a computer program for studying the geometry and topology of 3-manifolds, \url{http://snappy.computop.org}
\bibitem{BolotP13}
Bolotin D A and Poslavsky S V 2013 Introduction to Redberry: the computer algebra system designed for tensor manipulation {\it Preprint} arXiv:1302.1219
\bibitem{Redberry}
\url{http://redberry.cc/}

\end{thebibliography}

\end{document}
