\chapter{Tensor fields} \label{s:ten}

\minitoc

\section{Introduction}

Having presented vector fields in Chap.~\ref{s:vec}, we move now to more
general tensor fields. We keep the same example manifold, $M = \mathbb{S}^2$,
as in Chap.~\ref{s:man} and \ref{s:vec}.

\section{Differential forms}

Let us continue with the same example notebook as that considered in
Chap.~\ref{s:vec}. There, we had introduced $f$ as a scalar field on
the 2-dimensional manifold $M = \mathbb{S}^2$ (cf. Sec.~\ref{s:vec:tangent_impl}).
The differential of $f$ is a 1-form on $M$:
\jup{te80.png}
\jup{te81.png}
A 1-form is actually a tensor field of type $(0,1)$:
\jup{te82.png}
while a vector field is a tensor field of type $(1,0)$:
\jup{te83.png}
Specific 1-forms are those forming the dual basis (coframe) of a given vector
frame: for instance for the vector frame \code{eU} = $(\dert{}{x},\dert{}{y})$
on $U$, considered as a basis of the free $C^\infty(U)$-module $\mathfrak{X}(U)$,
we have:
\jup{te84.png}
\jup{te85.png}
Since \code{eU} is the default frame on $M$, the default display of $\mathrm{d}f$
is performed in terms of its coframe:
\jup{te86.png}
We may check that in this basis, the components of $\left. \mathrm{d}f \right| _U$
are nothing but the partial derivatives of the coordinate expression of $f$
with respect to coordinates $(x,y)$:
\jup{te87.png}
\jup{te88.png}
In the coframe associated with \code{eV} = $(\dert{}{x'},\dert{}{y'})$:
\jup{te89.png}
Since \code{eV} is not the default vector frame on $M$ and \code{XV} = $(V,(x',y'))$
is not the default chart on $M$, we get the individual components by
specifying both \code{eV} and \code{XV}, in addition to the index, in the
square-bracket operator:
\jup{te90.png}
We may then check that the components in the frame \code{eV}
are the partial derivatives with respect to the coordinates \code{xp} = $x'$ and
\code{yp} = $y'$ of the chart \code{XV}:
\jup{te91.png}
\jup{te92.png}
The parent of $\mathrm{d}f$ is the set $\Omega^1(M)$ of all 1-forms on $M$,
considered as a $C^\infty(M)$-module:
\jup{te93.png}
\jup{te94.png}
This module is actually the dual of the vector-field module $\mathfrak{X}(M)$,
which is represented
by the Python object \code{YM} (cf. Sec.~\ref{s:vec:vector_field_impl}):
\jup{te95.png}
Consequently, a 1-form acts on vector fields, yielding an element of
$C^\infty(M)$, i.e. a scalar field:
\jup{te96.png}
This scalar field is nothing but the result of the action of $\w{v}$ on $f$
(cf. Sec.~\ref{s:vec:action_on_scalar}):
\jup{te97.png}

\section{More general tensor fields}

We construct a tensor of type $(1,1)$ by taking the tensor product
$\w{v}\otimes \mathrm{d}f$:
\jup{te98.png}
\jup{te99.png}
\jup{te100.png}
We can use the method \code{display\_comp()} for a display component by
component:
\jup{te101.png}
The parent of $t$ is the set $\mathcal{T}^{(1,1)}(M)$ of all type-$(1,1)$
tensor fields on $M$,
considered as a $C^\infty(M)$-module:
\jup{te102.png}
\jup{te103.png}
As for vector fields, since $M$ is not parallelizable, the $C^\infty(M)$-module
$\mathcal{T}^{(1,1)}(M)$ is not free and the tensor fields are described by
their restrictions to parallelizable subdomains:
\jup{te104.png}
These restrictions form free modules:
\jup{te105.png}
\jup{te106.png}

\section{Riemannian metric}

\subsection{Defining a metric}

The standard metric on $M=\mathbb{S}^2$ is that induced by the Euclidean metric of $\mathbb{R}^3$. Let us start by defining the latter:
\jup{te107.png}
The metric $g$ on $M$ is the pullback of $h$ associated with the embedding $\Phi$
introduced in Sec.~\ref{s:vec:tangent_impl}:
\jup{te108.png}
Note that we could have defined $g$ intrinsically, i.e. by providing its components in the two vector frames \code{eU} and \code{eV}, as we did for the metric $h$ on $\mathbb{R}^3$. Instead, we have chosen to get it as the pullback by $\Phi$ of $h$, as an example of pullback associated with some differential map.

The metric is a symmetric tensor field of type (0,2):
\jup{te109.png}
The expression of the metric in terms of the default frame on $M$ (\code{eU}):
\jup{te110.png}
We may factorize the metric components:
\jup{te111.png}
\jup{te112.png}
A matrix view of the components of $g$ in the manifold's default frame:
\jup{te113.png}
Display in terms of the vector frame $(V, (\partial_{x'}, \partial_{y'}))$:
\jup{te114.png}
The metric acts on vector field pairs, resulting in a scalar field:
\jup{te115.png}
\jup{te116.png}
\jup{te117.png}

\subsection{Levi-Civita connection}

The Levi-Civita connection associated with the metric $g$ is
\jup{te118.png}
The nonzero Christoffel symbols of $g$ (skipping those that can be deduced by symmetry on the last two indices) w.r.t. the chart \code{XU}:
\jup{te119.png}
$\nabla_g$ acting on the vector field $\w{v}$:
\jup{te120.png}
\jup{te121.png}

\subsection{Curvature}

The Riemann curvature tensor of the metric $g$ is
\jup{te122.png}
The components of the Riemann tensor in the default frame on $M$ are
\jup{te123.png}
The parent of the Riemann tensor is the $C^\infty(M)$-module of
type-(1,3) tensor fields on $M$:
\jup{te124.png}
The Riemann tensor is antisymmetric on its two last indices (i.e. the indices
at position 2 and 3, the first index being at position 0):
\jup{te125.png}
The Riemann tensor of the Euclidean metric $h$ on $\mathbb{R}^3$ is identically zero,
i.e. $h$ is a flat metric:
\jup{te126.png}
The Ricci tensor is
\jup{te127.png}
while the Ricci scalar is
\jup{te128.png}
We recover the fact that $(\mathbb{S}^2,g)$ is a Riemannian manifold of constant positive curvature.

In dimension 2, the Riemann curvature tensor is entirely determined by the Ricci scalar $R$ according to
\be
 R^i_{\ \, jlk} = \frac{R}{2} \left( \delta^i_{\ \, k} g_{jl} - \delta^i_{\ \, l} g_{jk} \right)
\ee
Let us check this formula here, under the form
$R^i_{\ \, jlk} = -R g_{j[k} \delta^i_{\ \, l]}$:
\jup{te129.png}
Similarly the relation $\mathrm{Ric} = (R/2)\; g$ must hold:
\jup{te130.png}

\subsection{Volume form}

The \defin{volume form} (or \defin{Levi-Civita tensor}) associated with the
metric $g$ and for which the vector frame $(\partial_x,\partial_y)$ is
right-handed is the following 2-form
\jup{te131.png}
The exterior derivative of $\epsilon_g$ is a 3-form
\jup{te132.png}
Of course, since the dimension of $M$ is 2, all 3-forms vanish identically:
\jup{te133.png}











